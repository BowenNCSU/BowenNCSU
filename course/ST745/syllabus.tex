%Syllabus 3/3/2003
\documentclass[12pt]{article}
\usepackage[dvips]{graphicx}
\usepackage{amsthm}
\usepackage{amsmath}
\usepackage{lscape}
\usepackage{verbatim}

\textwidth=15.5cm \textheight=22.3cm \hoffset-1cm \voffset-1.4cm

\begin{document}
\renewcommand{\baselinestretch}{1.5}

\begin{center}
\large ST 745, Analysis of Survival Data, Spring 2011 \\
Tuesday $\&$ Thursday 1:30 - 2:45PM, SAS Hall, 1216
\end{center}

\bigskip
\noindent {\bf Instructor:} Dr. Wenbin Lu

\noindent Email: lu@stat.ncsu.edu

\noindent Homepage:
http://www4.stat.ncsu.edu/$\sim$lu/ST745/ST745.html

\noindent Office: 5212 SAS Hall

\noindent Phone: (919) 515-1915

\noindent Office Hours: Tuesday 3-4PM

\bigskip
\noindent {\bf Teaching Assistant:} Mingzhu Hu

\noindent Email: mhu4@ncsu.edu

\noindent Office Hours: Wednesday 10AM-12PM

\noindent Location: Tutorial Center in 1101 SAS Hall

\bigskip
\noindent {\bf Course Prerequisite:} ST 521, 522.

\bigskip
\noindent {\bf Course Resources:}
\begin{enumerate}
\item[$\bullet$] Text Book: Survival Analysis: Techniques for
Censored and Truncated Data, by John P. Klein and Melvin L.
Moeschberger (2003, 2nd Ed.)
\item[$\bullet$] References: The Statistical Analysis of Failure Time Data by John D.
Kalbfleisch and Ross L. Prentice (2002, 2nd Ed.); Modeling
Survival Data: Extending the Cox Model by Terry M. Therneau and
Patricia M. Grambsch (2000).
\item[$\bullet$] Software: SAS or R
\item[$\bullet$] Additional Online Materials: Please refer to my
homepage, teach section. The contents will be updated regularly.
\end{enumerate}

\medskip
\noindent {\bf Course Objectives:} By the end of this semester,
you need to know the following:
\begin{enumerate}
\item[$\bullet$] Concepts of censoring and truncation.
\item[$\bullet$] Basic quantities for describing the distribution of survival data.
\item[$\bullet$] Parametric methods for fitting the survival data.
\item[$\bullet$] Nonparametric estimation method (eg. Kaplan-Meier estimates).
\item[$\bullet$] Semiparametric method for regression problems (eg. Cox proportional hazards model).
\item[$\bullet$] Using SAS or R software to analyze survival data
\end{enumerate}

\medskip

\noindent {\bf Grading:} Homework 20\%, Mid-term 30\%, Final Exam
50\%


\end{document}
